Das Spiel soll mit der Tastatur und der Maus gesteuert werden k\"onnen. Für den\_die Spieler\_in, der\_die gerade an der Reihe ist,
einen Wurf zu parametrisieren, werden im GamePlayState zwei Input-Felder und ein Wurf-Button angezeigt. In eines der Eingabefelder
lässt sich ein Winkel als natürliche Zahl im Bereich von 0 bis 360 eingeben. In das andere Eingabefeld lässt sich die Geschwindigkeit ebenfalls als 
natürliche Zahl, aber im Bereich von 0 bis 200 eingeben. Andere Eingaben sollen jeweils nicht möglich sein. Damit der\_die Spieler\_in weiß, was er\_sie
in die Input-Felder eingeben soll, sind diese Felder beschriftet. So steht beispielsweise links neben dem einen Eingabefeld "Winkel:" und
links neben dem anderen Eingabefeld "Geschwindigkeit:".

Kann gerade kein Wurf parametrisiert werden, d. h. wenn der Wurf 
unterwegs ist, wird kein Eingabepanel (die Eingabefelder, deren Beschriftung und der Wurfbutton) angezeigt. 

Über die \textbf{Enter}-Taste oder über Klick auf den Wurf-Button wird der Wurf ausgelöst.

Das Framework stellt Operationen bereit, die Sie nutzen k\"onnen, um diese Funktionalit\"at zu implementieren.