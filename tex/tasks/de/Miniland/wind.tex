Das Wurfverhalten wird nun zusätzlich von Wind beeinflusst. Zu jeder Runde, also jeder neuen Karte/Skyline/Landschaft, wird zufällig festgelegt, wie stark der Wind von links oder rechts weht. Dies wird visualisiert (z. B. durch einen Windpfeil). Der Wind beeinflusst den Wurf realistisch.

Für einen Wurf von links nach rechts lässt sich die Position $(x, y)$ des Wurfobjekts in Abhängigkeit der Zeit $t$, der Startposition $(x_0, y_0)$ der Startgeschwindigkeit $v$ und des Anfangswinkels $\alpha$ nun folgendermaßen berechnen: 

\begin{align*}
v_x &= \cos (\alpha) \cdot v \\
v_y &= \sin (\alpha) \cdot v \\
x &= x_0 + (v_x \cdot t) + (\frac{1}{2} \cdot w_{scale} \cdot w \cdot t^2) \\
y &= y_0 - (v_y \cdot t) + (\frac{1}{2} \cdot g \cdot t^2) \\
\end{align*}

Die Variable $w$ besteht aus einem Wert zwischen -15 und 15. Diese Variable sollten Sie, analog zu der Skalierung des \textit{delta}
der Update-Methode von Slick, mit dem Faktor $w_{scale}$ so verändern, dass der Wurf realistisch vom Wind beeinflusst wird. 

Diesen "wind scaling"-Faktor müssen Sie im Testadapter wiederum über eine Methode bereitstellen, damit die Flugkurve exakt testbar wird.