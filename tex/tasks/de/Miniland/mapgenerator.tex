Wird im Hauptmen\"u auf den Button \glqq Level\grqq{} geklickt, so soll der Spieler eine Datei auswählen k\"onnen, welche eingelesen werden soll.
Anschlie"send soll das eingelesene Level intern gespeichert werden.
Dabei gibt es folgende \"Ubersetzung von Zeichen zu Objekt:

\begin{itemize}
\item M : Mario-Aufziehroboter
\item W : Wand
\item D : T\"ur
\item S : Stahltr\"ager-Sockel
\item X : Gefahr (Stacheln)
\item \_ : nichts (leeres Feld)
\end{itemize}

Zus\"atzlich soll jedes unbekannte Zeichen, welches hier nicht aufgelistet ist, als nichts bzw. leeres Feld interpretiert werden. 
Wichtiger Hinweis: das Schema wird in allen Ausbaustufen erweitert um neue Objekte mit neuen Zeichen.\\
Es soll auch sichergestellt sein, dass das eingelesene Level rechteckig ist. Das bedeutet, dass jede Zeile in der einzulesenden Textdatei gleich viele Zeichen besitzt.
Besteht eine Zeile aus \dq WWW\dq \,und die n\"achste aus \dq WWWW\dq \,so ist dies nicht erlaubt und die entsprechende Datei soll nicht gelesen werden.
Beispieldateien befinden sich in der Vorlage im Ordner \glqq level\grqq{}.\\
