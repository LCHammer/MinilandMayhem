Neben den normalen, offenen T"uren soll es nun auch verschlossene T"uren geben.
L"auft ein Mario gegen solch eine verschlossene T"ur, so geht er nicht hindurch, sondern prallt ab und "andert seine Laufrichtung, wie als w"are er gegen eine Wand gelaufen.
Um eine Verschlossene T"ur zu aufzuschlie"sen, muss zuerst ein Schl"ussel (in Form eines neuen sammelbaren Gegenstandes) eingesammelt werden. Hat ein Mario dies getan und ber"uhrt er nun eine verschlossene T"ur, so wird diese zu einer normalen T"ur und der Mario wird vom Spiel entfernt. 
Auch andere Marios k"onnen nun durch die aufgeschlossene T"ur gehen, die verschlossene T"ur verh"alt sich dann identisch zu einer normalen T"ur.

Beachten Sie, dass nur der Mario, welcher auch den Schl"ussel eingesammelt hat, auch derjenige ist, welcher die T"ur "offnet und kein anderer! 
Ausnahme: Der andere Mario hat einen anderen Schl"ussel eingesammelt. Jeder Shl"ussel passt in jede verschlossene T"ur.\\
Die Spielobjekte Schl"ussel und verschlossene T"ur sollen auch in den Übersetzungsschema des Levelgenerators eingef"ugt werden. Dabei soll das Zeichen "K" (eng. Key) einen Schl"ussel erstellen und ein kleines "d" \: eine verschlossene T"ur. 