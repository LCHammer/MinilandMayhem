Wird ein Spiel beendet, soll es im Endscreen einen neuen Button geben mit der Aufschrigt \glqq Highscore Speichern\grqq.
Wird dieser Knopf gedr"uckt, so soll eine neue Textdatei im Ordner \texttt{highscores} erstellt werden oder die bereits bestehende Datei angepasst werden. 
Die neue Textdatei soll folgendes Namensschema bestzen: "Highscore\_levelname.txt". Ist der Levelname beispielsweise \glqq Level1\grqq, so soll die Highscoredatei \glqq Highscore\_Level1.txt\grqq hei"sen.
In die Datei soll nun ein neuer Eintrag geschrieben werden, welcher die erreichte Wertung repr"asentiert. 

Der Aufbau des Eintrags ist folgender: \glqq p,e/i\grqq, wobei e für die Anzahl Marios steht, die erfolgreich durch eine T"ur gelaufen sind, i steht f"ur die Gesamtzahl der Marios des Levels und p steht für die erzielte Punktzahl.
Haben es also 3 von 5 Marios durch die T"ur geschafft und die resultierende Punktzahl ist 1234, so soll der neue Eintrag \glqq 1234,3/5\grqq sein.
Eine Zeile in der Textdatei darf nur einen Eintrag enthalten.
Au"serdem sollen in der Datei nur die f"unf besten Eintr"age zu sehen sein.
Ein Eintrag gilt als besser als ein anderer, wenn mehr Punkte erzielt wurden oder bei gleicher Punktzahl mehr Marios durch eine T"ur gelaufen sind.