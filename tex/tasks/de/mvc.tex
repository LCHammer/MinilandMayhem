Wenn Sie das vorgegebene \emph{eea}-Framework verwenden, dann achten Sie bei Ihrer Implementierung auf die sinnvolle Verwendung von Entitäten, Ereignissen und Aktionen. Verwenden Sie Packages zur Strukturierung Ihres Codes. Ihr\_e Tutor\_in wird im Rahmen des Code-Review bewerten, wie gut Ihnen die Umsetzung gelungen ist.

Andernfalls strukturieren Sie Ihr Programm nach dem Prinzip \emph{MVC (\glqq Model View Controller\grqq)}, siehe T20.42-47. F\"uhren Sie eine sinnvolle Einteilung in Pakete (\emph{package hierarchy}) ein. Erstellen Sie ein Klassendiagramm in UML, welches die Struktur des Programms wiedergibt, und markieren Sie darin jeweils die Klassen der Bereiche Model, View
und Controller. Ihr\_e Tutor\_in wird sich das Klassendiagramm ansehen und ggf. Fragen hierzu stellen. Insbesondere beim Code-Review wird auf die Umsetzung des MVC-Prinzips geachtet.
