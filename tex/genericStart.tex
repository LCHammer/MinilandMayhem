\usepackage{latexsym,ifthen,xspace}
\usepackage{amsfonts}
\usepackage{amssymb}
\usepackage{lab}
\usepackage{graphicx,listings}
%\usepackage{ulem}								  %fuer strike out
\usepackage{color}								%TODO remove this!
\usepackage[
bookmarks=true,									% Lesezeichen erzeugen
bookmarksopen=true,								% Lesezeichen ausgeklappt
bookmarksnumbered=true,							% Anzeige der Kapitelzahlen am Anfang der Namen der Lesezeichen
%pdfstartpage=Zahl, 
												% Seite, welche automatisch ge\"offnet werden soll
												% praktisch, wenn z.B. im Inhaltsverzeichnis gestartet
												% werden soll oder eine Seite bearbeitet wird.
%baseurl=http://www.server.de/dateiname.pdf,		% URL des PDF-Dokuments (oder Hintergrundinformationen)
pdftitle={Projekt zur GdI 1},	% Autor  des PDF-Dokuments
%pdfsubject={Kurzbeschreibung als ein Satz},		% Inhaltsbeschreibung des PDF-Dokuments
%pdfkeywords={Stichw\"orter},						% Stichwortangabe zum PDF-Dokument
%breaklinks=true,								% erm\"oglicht einen Umbruch von URLs
%colorlinks=true,								% Einf\"arbung von Links
%linkcolor=black,								% Linkfarbe: schwarz
%anchorcolor=black,								% Ankerfarbe: schwarz
%citecolor=black,								% Literaturlinks: schwarz
%filecolor=black,								% Links zu lokalen Dateien: schwarz
%menucolor=black,								% Acrobat Men\"u Eintr\"age: schwarz
%pagecolor=black,								% Links zu anderen Seiten im Text: schwarz
%urlcolor=black    								% URL-Farbe: schwarz
] {hyperref}
\lstset{language=Java}
\oddsidemargin 6pt \evensidemargin 6pt \marginparwidth 48pt \marginparsep 10pt 
\topmargin -18pt \headheight 12pt \headsep 25pt \footskip 30pt 
\textheight 625pt \textwidth 431pt \columnsep 10pt \columnseprule 0pt
\parindent0pt 

\newcommand{\registerExamDE}{%
Zur Teilnahme am Projekt m�ssen sich \emph{alle} Gruppenmitglieder vor dem Projekt im 
Webreg-System f�r das Projekt angemeldet haben und der Gruppe zugeteilt worden sein. 
Im Webreg-System ist auch die Anmeldung direkt als Gruppe m�glich und wird auch
\textbf{angeraten}, damit die Gruppen \emph{nicht zuf�llig aufgef�llt werden m�ssen}.}
\newcommand{\registerExamEN}{...}
\newcommand{\registerSLDE}{%
Zur Teilnahme am Projekt m\"ussen sich \emph{alle} Mitglieder einer gegebenen Projektgruppe im Portal
im extra f\"ur das Projekt angelegten Kurs angemeldet haben und der gleichen Projektgruppe 
beigetreten sein. Wir raten \textbf{dringend} dazu, sich m\"oglichst fr\"uh in einer 
gemeinsamen Projektgruppe anzumelden. Bitte beachten Sie, dass bei der Eintragung in die Projektgruppe
nur Studierende einer Projektgruppe beitreten k\"onnen, die bereits im Kurs zum Projekt
\emph{angemeldet} sind \emph{und noch keiner Projektgruppe beigetreten sind}. Gruppen, die am Ende
des Anmeldeintervalls noch nicht die Gr\"o\ss{}e 4 haben, werden von uns mit zuf\"allig gew\"ahlten
anderen Studierenden aufgef\"ullt. Bitte versuchen Sie dies nach M\"oglichkeit zu vermeiden!}
\newcommand{\registerSLEN}{To take part in the project, \emph{all} members of a given project group
have to be registered for the course offered for the project in the Moodle portal, and must have joined
the same project grop. Moodle also supports the registration as a group, we \emph{strogly encourage you
to use this}! Please note that you can only select students for your lab group who have already
\emph{joined the course} and \emph{not yet joined a project group}. Groups that end up with less than
four members will be filled by us with other randomly chosen students. Please try to avoid this
if at all possible!}
\newcommand{\myRegister}{\registerSLDE}

\newcommand{\afterExamDE}{direkt nach Ablauf der Abmeldefrist von der Pr\"ufung\xspace}
\newcommand{\afterExamEN}{direkt nach Ablauf der Abmeldefrist von der Pr\"ufung\xspace}
\newcommand{\fourBeforeDE}{vier Wochen vor Beginn der Projektphase\xspace}
\newcommand{\fourBeforeEN}{four weeks before the official start of the project\xspace}
\newcommand{\myAvailable}{\fourBeforeDE}

\newcommand{\portalDE}{Lernportal\xspace}
\newcommand{\portalEN}{portal\xspace}
\newcommand{\webregDE}{Webreg\xspace}
\newcommand{\webregEN}{Webreg\xspace}
\newcommand{\regSystem}{\portalDE}

\newcommand{\initInInitDE}{innerhalb der Methode \verb!init()!\xspace}
\newcommand{\initInInitEN}{inside the methode \verb!init()!\xspace}
\newcommand{\initInConstructorDE}{im Konstruktor\xspace}
\newcommand{\initInConstructorEN}{inside the constructor\xspace}
\newcommand{\initWhere}{\initInInitDE}
